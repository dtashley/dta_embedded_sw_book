%%Sets of real numbers.
\newcommand{\vworkrealset}{{\mathbb{R}}}
\newcommand{\vworkrealsetnonneg}{{\mathbb{R}^+}}
%
%%Sets of integers.
\newcommand{\vworkintset}{{\mathbb{Z}}}
\newcommand{\vworkintsetnonneg}{{\mathbb{Z}^+}}
\newcommand{\vworkintsetpos}{{\mathbb{N}}}
%
%%Sets of rational numbers.
\newcommand{\vworkratset}{{\mathbb{Q}}}
\newcommand{\vworkratsetnonneg}{{\mathbb{Q}^+}}
%
%%Sets of irrational numbers.
\newcommand{\vworkirratset}{{\mathbb{error}}}
\newcommand{\vworkirratsetnonneg}{{\mathbb{error}^+}}
%
%%"Divides" and "Not Divides".  Am not able to find quite
%%the right symbol for "Not Divides" at this time.
\newcommand{\vworkdivides}{\mid}
\newcommand{\vworknotdivides}{\hspace{-0.125em}\not\hspace{0.245em}\mid\hspace{0.135em}}
%%
%%The implication operator, which may change throughout the work.  Both a horizontal
%%and vertical form are defined.
\newcommand{\vworkhimp}{\to}
\newcommand{\vworkvimp}{\downarrow}
%%
%%The symbol for logical equivalence.  There are three forms defined, the standard,
%%the long, and the short, which may be identical.
\newcommand{\vworkequiv}{\leftrightarrow}
\newcommand{\vworkshortequiv}{\leftrightarrow}
\newcommand{\vworklongequiv}{\longleftrightarrow}
\newcommand{\vworkvertequiv}{\updownarrow}
%
% End of ucbka_mdefs.tex

